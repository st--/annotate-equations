\documentclass[xcolor={dvipsnames,rgb},aspectratio=1610]{beamer}
% xcolor: e.g. provides \colorlet
% dvipsnames option: extra named colours
% aspectratio=1610: for modern widescreen monitors...
% see https://www.overleaf.com/learn/latex/Using_colours_in_LaTeX for more detail

% equation annotations also work in articles, of course (then might need explicit \usepackage[dvipsnames]{xcolor}

%%% standard math packages for equations:
\usepackage{amsmath}
\usepackage{amssymb}
\usepackage{mathtools}

%%% tikz & libraries
\usepackage{tikz}
\usetikzlibrary{backgrounds}
\usetikzlibrary{arrows,shapes}
\usetikzlibrary{tikzmark} % for \tikzmarknode
\usetikzlibrary{calc} % for computing the midpoint between two nodes, e.g. at ($(p1.north)!0.5!(p2.north)$) 

% Commands for Highlighting text -- non tikz method
\newcommand{\highlight}[2]{\colorbox{#1!17}{$#2$}}
\newcommand{\highlightdark}[2]{\colorbox{#1!47}{$#2$}}
%%% NOTE: \colorbox sets the second argument in text mode, so for use within equations we wrap it in $ $ again
% if you use \highlight or \highlightdark in subscripts, you need to pass \scriptstyle to get the font size right
% e.g. $ \mathbb{E}_{\highlight{BurntOrange}{\scriptstyle y}} $

\begin{document}

\begin{frame}{Runtime complexity}
    \LARGE %%% increase font size to make equation more readable!

    %%% we define our own colors upfront - this makes it easier to keep it consistent if you change your mind
    \colorlet{colorp}{NavyBlue}
    \colorlet{colorT}{WildStrawberry}
    \colorlet{colork}{OliveGreen}
    \colorlet{colorM}{RoyalPurple}
    \colorlet{colorNb}{Plum}
    \colorlet{colorIs}{black}

\onslide<+->{ % the very first <+-> only increments the counter, it will already be visible on first page
    \begin{equation*}
        \mathcal{O}\big(
            (
            % \tikzmarknode is what links parts of the equation and corresponding annotations
            \tikzmarknode{p1}{\highlight{colorp}{p}}
            \tikzmarknode{k1}{\highlight{colork}{\kappa}}^3 % note that we have the ^3 outside the \tikzmarknode
            )
            \tikzmarknode{T1}{\highlight{colorT}{T}}
            +
            (
            \tikzmarknode{p2}{\highlight{colorp}{p}} % tikzmarks need distinct names!
            \tikzmarknode{k2}{\highlight{colork}{\kappa}}
            )
            (
            \tikzmarknode{T2}{\highlight{colorT}{T}}^2
            \tikzmarknode{Is}{{\color{colorIs}|\mathcal{I}^*|}}
            \tikzmarknode{Nb}{\highlight{colorNb}{N_b}}
            \tikzmarknode{M}{\highlight{colorM}{M}}
            )
        \big)
    \end{equation*}
}
    \begin{tikzpicture}[overlay,remember picture,>=stealth,nodes={align=left,inner ysep=1pt},<-]
        % manually adjust \node's yshift until it's at appropriate distance above the equation
        % can add xshift as well, if needed (also positive or negative)
\onslide<+->{
        % for "p":
        % default anchor is at center
        \node[color=colorp!85,yshift=0.9cm] (ptext)
            at ($(p1.north)!0.5!(p2.north)$)  % centered between p1 and p2
            {\textsf{\footnotesize \# of nodes}};
    
        % double arrow to two uses within the equation:
        \draw [<->,color=colorp] (p1.north) |- (ptext.south) -| (p2.north);
}
\onslide<+->{
        % for "|I*|"
        % anchor=west: align left edge of text on top of tikzmark in equation
        % could also use anchor=north west or anchor=south west, and adjust yshift to make up for it
        \node[anchor=west,color=colorIs!85,yshift=3em] (Istext) at (Is.north)
            {\textsf{\footnotesize size of set of allowed interventions}};
        \draw [color=colorIs](Is.north) |- ([xshift=-0.3ex,yshift=-0.2ex]Istext.south east);
            % arrow from the equation; this one is above the equation: start at north anchor of tikzmark
            % - south east: we want line to end at bottom right of text
            % - negative xshift makes it a little bit shorter
            % - negative yshift: manual adjustment for aesthetics
}
\onslide<+->{
        % for "N_b"
        \node[anchor=north west,color=colorNb!85,yshift=1.5em] (Nbtext) at (Nb.north)
            {\textsf{\footnotesize \# of samples per batch}};
        \draw [color=colorNb](Nb.north) |- ([xshift=-0.3ex]Nbtext.south east); % south east: we want line to end at bottom right of text; negative xshift makes it a little bit shorter
}
\onslide<+->{
        % for "T"
        % the xshift here is a manual adjustment so that the space between two words falls nicely on top of the line to the second \kappa
        \node[anchor=north,color=colorT!85,yshift=-0.6cm,xshift=0.2ex] (Ttext) at ($(T1.south)!0.5!(T2.south)$)
            {\textsf{\footnotesize \# of graphs in $\hat{\mathcal{G}}_T$}};
        \draw [<->,color=colorT] (T1.south) |- (Ttext.south) -| (T2.south);
}
\onslide<+->{
        % for "\kappa"
        \node[anchor=north,color=colork!85,yshift=-1.6cm] (ktext) at ($(k1.south)!0.5!(k2.south)$)
            {\textsf{\footnotesize max.\ indegree in $\hat{\mathcal{G}}_T$}};
        \draw [<->,color=colork] (k1.south) |- (ktext.south) -| (k2.south);
}
\onslide<+->{
        % for "M"
        % a different way of getting the node in the right place:
        \path (M.south) ++ (0,-3.5em) node[anchor=north west,color=colorM!85] (Mtext)
            {\textsf{\footnotesize \# of samples for $\mathbb{E}_y$}};
        \draw [color=colorM] (M.south) |- ([xshift=-0.3ex]Mtext.south east);
}
    \end{tikzpicture}
\end{frame}

\end{document}
