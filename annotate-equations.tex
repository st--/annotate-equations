\documentclass{article}
\usepackage[a4paper, margin=1in]{geometry}

\usepackage[dvipsnames]{xcolor}

%%% standard math packages for equations:
\usepackage{amsmath}
\usepackage{amssymb}
\usepackage{mathtools}

\usepackage{hyperref}
\usepackage{cleveref}

\usepackage{annotate-equations}

%\usepackage{showexpl} % LTXexample
\usepackage[skins,listings]{tcolorbox}
\newtcblisting{LTXexample}[1][]{%
  colframe=black!20,
  colback=black!10,
  coltitle=red!50!yellow!3!white,
  bicolor,colbacklower=white,
  fonttitle=\sffamily\bfseries,
%  listing options={escapeinside=\`\`},
  %sidebyside,
  text above listing,
  #1}

\title{\texttt{annotate-equations.sty}, v.0.2.1}
\author{ST John}
\date{\url{https://github.com/st--/annotate-equations}}


\begin{document}

\maketitle



%\section{Introduction}

This package is there to make it easier to make annotated equations in \LaTeX, such as in this example:
\begin{LTXexample}[]
\vspace{4em}
\renewcommand{\eqnhighlightheight}{\vphantom{\hat{H}}\mathstrut}
\begin{equation*}
    i \tikzmarknode{hbar}{\mathstrut\hbar} \frac{\partial}{\partial t} \eqnmarkbox[blue]{Psi1}{\Psi(x, t)} = \eqnmark[red]{Hhat}{\hat{H}} \eqnmarkbox[blue]{Psi2}{\Psi(x, t)}
\end{equation*}
\annotate[yshift=3em]{above}{hbar}{$\hbar = \frac{h}{2\pi}$, reduced Planck constant}
\annotate[yshift=1em]{above}{Hhat}{Hamilton operator}
\annotatetwo[yshift=-1em]{below}{Psi1}{Psi2}{Wave function}
\vspace{1em}
\end{LTXexample}
There is still a bit of manual tweaking required (such as adding vertical space before/after the equation), but hopefully this package will already make it a bit more inviting to annotate your equations!

Note that this package relies on TikZ's \texttt{remember picture} option and therefore you have to compile your \LaTeX{} document at least twice to get everything in the right place (or just use \texttt{latexmk}!).


\tableofcontents


\section{Marking annotation targets within your equation}

\newcommand{\cmdoption}[1]{$\langle$\textit{#1}$\rangle$}

Use
\verb|\eqnmarkbox[|\cmdoption{color}\verb|]{|\cmdoption{node name}\verb|}{|\cmdoption{equation term(s)}\verb|}|
or
\verb|\eqnmark[|\cmdoption{color}\verb|]{|\cmdoption{node name}\verb|}{| \cmdoption{equation term(s)}\verb|}|
to define the target of an annotation within your equation. \verb|\eqnmarkbox| adds background shading, whereas \verb|\eqnmark| changes the text color.
(You can also use
\verb|\tikzmarknode{|\cmdoption{node name}\verb|}{|\cmdoption{equation term(s)}\verb|}|,
but this is likely to end up with the arrow tip too close to the target, so you may want to also pass the \texttt{outer ysep} option, see \cref{sec:outerysep}.)
%
\begin{LTXexample}[text outside listing,lefthand width=1in]
\begin{equation*}
    \eqnmarkbox[blue]{node1}{e_q^n}
    \eqnmark[red]{node2}{f(x)}
    \tikzmarknode{node3}{kT}
\end{equation*}
\end{LTXexample}
\noindent


\section{Simple annotations}
\label{sec:annotate}

Once you have defined nodes within your equations, you can annotate them using
\verb|\annotate[|\cmdoption{tikz options}\verb|]{|\cmdoption{annotate keys}\verb|}{|\cmdoption{node name[,\dots]}\verb|}{|\cmdoption{annotation text}\verb|}|.
%
\cmdoption{tikz options} is passed through to the options for the TikZ node defining the annotation; its most important use is to set the \texttt{yshift}.
For \cmdoption{annotate keys}, see \cref{sec:annotate-keys}.
\cmdoption{node name} is the same name you used to mark the node within the equation, e.g.\ using \verb|\eqnmarkbox|.
\cmdoption{annotation text} is the text of the annotation itself.
%
\begin{LTXexample}[text outside listing,lefthand width=1in]
\begin{equation*}
    \eqnmarkbox[blue]{node1}{e_q^n}
    \eqnmark[red]{node2}{f(x)}
    \tikzmarknode{node3}{kT}
\end{equation*}
\annotate[yshift=1em]{}{node1,node2}{my annotation text}
\end{LTXexample}
\noindent
%
\label{sec:tikz-options}
You generally need to manually adjust the \texttt{yshift} to move the annotations to an appropriate distance above (or negative values for below) the equation. If you want an annotation below the equation, with negative \texttt{yshift}, remember to also pass the \texttt{below} option (see \cref{sec:annotate-keys}).
(You can also adjust \texttt{xshift} if needed, also positive or negative.)

The annotation picks the same text color as given to \verb|\eqnmarkbox| or \verb|\eqnmark|, but you can also override it using \texttt{color} option.

One annotation can point to multiple targets, and multiple annotations can point to the same target.

\subsection{Annotation options}
\label{sec:annotate-keys}

\cmdoption{annotate keys} can be empty, or contain one or more of:
\begin{itemize}
    \item \texttt{above} (default) or \texttt{below},
    \item \texttt{right} (default) or \texttt{left},
    \item \texttt{label above} (default) or \texttt{label below}.

        Note: currently only works for \verb|\annotatetwo| (\cref{sec:annotatetwo}).
\end{itemize}
%
\begin{LTXexample}[text outside listing,lefthand width=1in]
\begin{equation*}
    \eqnmarkbox[blue]{node1}{e_q^n}
    \eqnmark[red]{node2}{f(x)}
    \tikzmarknode{node3}{kT}
\end{equation*}
\annotate[yshift=1em]{left}{node1}{my}
\annotate[yshift=-0.5em]{below,left}{node2}{annotation}
\annotate[yshift=-1em]{below}{node3}{text}
\end{LTXexample}
\noindent
%


\section{Double annotations}
\label{sec:annotatetwo}

\verb|\annotatetwo[|\cmdoption{tikz options}\verb|]{|\cmdoption{annotate keys}\verb|}{|\cmdoption{first node name}\verb|}{|\cmdoption{second node name}\verb|}{|\cmdoption{annotation text}\verb|}|.
\cmdoption{tikz options} and \cmdoption{annotate keys} are as described above in \cref{sec:tikz-options,sec:annotate-keys}. Note that \cmdoption{annotate keys} \texttt{left}/\texttt{right} have no effect in \verb|\annotatetwo|.

\begin{LTXexample}[text outside listing,lefthand width=1in]
\begin{equation*}
    \eqnmarkbox[red]{a1}{a} \eqnmarkbox[blue]{b1}{b} = \eqnmarkbox[green]{b2}{b} \eqnmarkbox{a2}{a}
\end{equation*}
\annotatetwo[yshift=1.5em]{above, label below}{a1}{a2}{var 1}
\annotatetwo[yshift=0.5em]{above}{b1}{b2}{var 2}
\annotatetwo[yshift=-0.5em]{below}{b2}{b1}{var 2}
\end{LTXexample}
\noindent
%
Color is picked from the first of the two nodes.


\section{Package options}

\subsection{Size of highlight: shrink to content or always full height}

\verb|\eqnhighlightheight| is inserted into every \verb|\eqnhighlight|, \verb|\eqncolor|, \verb|\eqnmark|, and \verb|\eqnmarkbox| and by redefining it you can specify the minimum height for the corresponding box:
%
\begin{LTXexample}[text outside listing,lefthand width=0.5in]
\renewcommand{\eqnhighlightheight}{}  % package default

\begin{equation*}
    \eqnmarkbox[red]{hbar}{\hbar} \eqnmarkbox[blue]{q}{q}
\end{equation*}
\end{LTXexample}
\noindent
%
\begin{LTXexample}[text outside listing,lefthand width=0.5in]
\renewcommand{\eqnhighlightheight}{\mathstrut}  % 0-width "constant" height

\begin{equation*}
    \eqnmarkbox[red]{hbar}{\hbar} \eqnmarkbox[blue]{q}{q}
\end{equation*}
\end{LTXexample}
\noindent
%
\verb|\eqnhighlightheight| is used in math mode.


Note that in some cases \verb|\mathstrut| might not be enough, as in the introductory example:
%
\begin{LTXexample}[text outside listing,lefthand width=0.5in]
\renewcommand{\eqnhighlightheight}{\mathstrut}  % 0-width "constant" height

\begin{equation*}
    \eqnmarkbox[red]{Hhat}{\hat{H}} \eqnmarkbox[blue]{Psi}{\Psi}
\end{equation*}
\end{LTXexample}
\noindent
%
You can create custom 0-width characters using \verb|\vphantom|:
%
\begin{LTXexample}[text outside listing,lefthand width=0.5in]
\renewcommand{\eqnhighlightheight}{\vphantom{\hat{H}}\mathstrut}  % custom 0-width height

\begin{equation*}
    \eqnmarkbox[red]{Hhat}{\hat{H}} \eqnmarkbox[blue]{Psi}{\Psi}
\end{equation*}
\end{LTXexample}
\noindent
%
(It looks more balanced if you still include the \verb|\mathstrut|.)

\subsection{Amount of shading of mark highlight}

\verb|\eqnhighlightshade| defines the percentage of the specified color to take:
%
\begin{LTXexample}[text outside listing,lefthand width=0.5in]
\renewcommand{\eqnhighlightshade}{17}  % package default

\begin{equation*}
    \eqnmarkbox[red]{hbar}{\hbar} \eqnmarkbox[blue]{q}{q}
\end{equation*}
\end{LTXexample}
\noindent
%
By redefining this command, you can change the ``alpha'' value of the highlight:
%
\begin{LTXexample}[text outside listing,lefthand width=0.5in]
\renewcommand{\eqnhighlightshade}{47}  % 0 is white, 100 is solid color

\begin{equation*}
    \eqnmarkbox[red]{hbar}{\hbar} \eqnmarkbox[blue]{q}{q}
\end{equation*}
\end{LTXexample}
\noindent

\subsection{Default formatting of annotation labels}
\label{sec:eqnannotationfont}

\verb|\eqnannotationfont| sets the \texttt{font} field of the TikZ annotation label and can be re-set to change its formatting:
\begin{LTXexample}[text outside listing,lefthand width=0.5in]
\renewcommand{\eqnannotationfont}{\sffamily\footnotesize}  % package default

\begin{equation*}
    \eqnmarkbox[blue]{v}{v}
\end{equation*}
\annotate[yshift=-0.5em]{below}{v}{velocity}
\vspace{1em}
\end{LTXexample}
\noindent
%
\begin{LTXexample}[text outside listing,lefthand width=0.5in]
\renewcommand{\eqnannotationfont}{\bfseries\small}

\begin{equation*}
    \eqnmarkbox[blue]{v}{v}
\end{equation*}
\annotate[yshift=-0.5em]{below}{v}{velocity}
\vspace{1em}
\end{LTXexample}
\noindent
%
Alternatively, you can also change the style of \verb|annotate equations/text|:
\begin{LTXexample}[text outside listing,lefthand width=0.5in]
\tikzset{annotate equations/text/.style={font=\bfseries\small}}

\begin{equation*}
    \eqnmarkbox[blue]{v}{v}
\end{equation*}
\annotate[yshift=-0.5em]{below}{v}{velocity}
\vspace{1em}
\end{LTXexample}
\noindent

\verb|\eqnannotationstrut| is defined to be a strut (zero-width height) to
provide minimum distance between the text and the corresponding arrow line. By
default it is \verb|\strut|, which has a similar effect to
\verb|\mathstrut| in \verb|\eqnhighlightheight|.

\begin{LTXexample}[text outside listing,lefthand width=0.5in]
\renewcommand{\eqnannotationstrut}{\strut}  % package default

\begin{equation*}
	\eqnmarkbox[blue]{size}{s}
\end{equation*}
\annotate[yshift=-0.5em]{below}{size}{The size}
\vspace{1em}
\end{LTXexample}
\noindent
%
\begin{LTXexample}[text outside listing,lefthand width=0.5in]
\renewcommand{\eqnannotationstrut}{}

\begin{equation*}
  \eqnmarkbox[blue]{size}{s}
\end{equation*}
\annotate[yshift=-0.5em]{below}{size}{The size}
\vspace{1em}
\end{LTXexample}


\subsection{Customize style}

You can change the style of the annotation arrow line by setting the style of \verb|annotate equations/arrow|:
\begin{LTXexample}[text outside listing,lefthand width=0.5in]
\tikzset{annotate equations/arrow/.style={color=ForestGreen, >=latex', very thick, dashed}}

\begin{equation*}
  \eqnmarkbox[blue]{size}{s} = \eqnmarkbox[red]{other}{x}
\end{equation*}
\annotate[yshift=-0.5em]{below}{size}{The size}
\annotatetwo[yshift=1em]{above}{size}{other}{the same}
\end{LTXexample}
\noindent
Note that it applies to all \verb|\annotate| and \verb|\annotatetwo| arrows within the scope.

You can also use this to change the arrow direction:
\begin{LTXexample}[text outside listing,lefthand width=0.5in]
\begin{equation*}
  \eqnmarkbox[blue]{size}{s} = \eqnmarkbox[red]{other}{x}
\end{equation*}
\tikzset{annotate equations/arrow/.style={->}}
\annotatetwo[yshift=1em]{above}{size}{other}{one and}
\tikzset{annotate equations/arrow/.style={<-}}
\annotatetwo[yshift=-1em]{below, label below}{size}{other}{the same}  % note that the "direction" of the arrow is from first to second mark
\end{LTXexample}

\section{Recommendations, tips \& tricks}

\subsection{Use \texttt{\textbackslash{}colorlet} for consistent, easily changeable colors}


\subsection{Relations such as ``$=$''}
\label{sec:mathrel}

Wrapping a mathematical relation symbol such as {\color{blue}$=$} in, for example, \texttt{\textbackslash{}tikzmarknode}, breaks how \TeX{} determines spacing in equations:
\begin{LTXexample}[text outside listing,lefthand width=0.5in]
\[
a \tikzmarknode{node1}{=} b
\]
\annotate[yshift=-1em]{below}{node1}{equality}
\end{LTXexample}
\noindent
This can be fixed by wrapping the \texttt{\textbackslash{}tikzmarknode} in \texttt{\textbackslash{}mathrel}:
\begin{LTXexample}[text outside listing,lefthand width=0.5in]
\[
a \mathrel{\tikzmarknode{node1}{=}} b
\]
\annotate[yshift=-1em]{below}{node1}{equality}
\end{LTXexample}

\subsection{Extra spacing between \texttt{\textbackslash{}tikzmarknode} and arrow}
\label{sec:outerysep}

If you want more space between arrow tip and annotated term, you can pass the \texttt{outer ysep} option to \texttt{\textbackslash{}tikzmarknode}:
\begin{LTXexample}[text outside listing,lefthand width=0.5in]
\[
a \mathrel{\tikzmarknode[outer ysep=5pt]{node1}{=}} b
\]
\annotate[yshift=-1em]{below}{node1}{equality}
\end{LTXexample}



\section{Known issues}

\begin{itemize}
    \item \texttt{label above}/\texttt{label below} is not implemented for \verb|\annotate|.

    \item Annotations of mathematical relations require some manual patching to get the correct surrounding spacing (see \cref{sec:mathrel}).

\end{itemize}


\section{Backwards-incompatible changes}

\subsection*{v0.2.0}

\subsubsection*{\texttt{\textbackslash{}eqnannotationtext} removed}

To make it easier to format multiline annotations, version 0.2.0 introduced the \verb|\eqnannotationfont| and \verb|\eqnannotationstrut| (zero-argument) commands (see \cref{sec:eqnannotationfont}).

In exchange, the \verb|\eqnannotationtext| (one-argument) command was removed. To upgrade, replace for example
\begin{tcblisting}{listing only}
    \renewcommand{\eqnannotationtext}[1]{\sffamily\tiny#1\strut}
\end{tcblisting}
with
\begin{tcblisting}{listing only}
    \renewcommand{\eqnannotationfont}{\sffamily\tiny}
    \renewcommand{\eqnannotationstrut}{\strut}
\end{tcblisting}

\end{document}
